\chapter{Fallstudie}

Nachdem die verschiedenen Konsistenzregeln aufgestellt und implementiert wurden, soll nun die Benutzbarkeit und die Erweiterbarkeit des Tools gezeigt werden.
Dafür wird in diesem Kapitel eine beispielhafte Anwendung des Tools aus sicht des Anwenders durgeführt (Abschnitt 1) und anschließend eine neue Regeln aus sicht des Entwicklers implementiert (Abschnitt 2).

\section{Anwendung am Beispiel einer Pizzabestellung}


Dafür wird der Vorgang einer Pizzabestellung betrachtet und mit Hilfe des Tools evaluiert.
Die benutzten BPMN- und BROS-Modelle basieren auf den Modellen von \cite{Schoen}.

\section{Erweiterbarkeit des Ansatzes}

Bisher wurden nur Regeln vorgestellt, die die Konsistenz von BPMN-Modellen zu BROS-Modellen prüfen.
Da das BROS-Modell gegenüber dem BPMN-Modell angereichert werden kann ist dies auch die häufigste Anwendung.
Allerdings können auch einige Regeln in die andere Richtung überprüft werden.
Um gleichzeitig die Erweiterbarkeit des Ansatzes und der Implementierung zu zeigen wird \textbf{Regel 6} aus \cref{sec:Konsistenzregeln} hinzugefügt.
Die Regel 6 überprüft das zu jedem BROS-Event und BROS-ReturnEvent ein dazugehöriges BPMN-Element existiert. 
Dabei können die BPMN Elemente von den Typen BPMN-Activity, BPMN-Gateway und BPMN-Event seien. 
Der genaue Typ des BPMN-Events ist dabei nicht relevant, es kann sich dabei unter anderem um ein BPMN-StartEvent oder auch ein BPMN-TerminationEvent handeln.

Jede Regel sollte zur besseren Übersicht in eine eigene Datei ausgelagert werden.
Die bereits bestehenden Regel befinden sich im Package \emph{io.framed.modules}.
Für die Regel 6 wird dafür die Datei \emph{Rule6BrosEvents.kt} erstellt.
Jede Datei enthält eine Funktion die einmalig zum Setup der Regeln ausgeführt wird.
Dies hat die Signatur \emph{fun Context.setupRule6BrosEvents()}.
Damit wird ein Parameter definiert, der den Typ \emph{Context} hat und als Receiver (\emph{this}) genutzt werden kann.
Innerhalb dieser Funktion können nun die Matching- und Verifikations-Funktionen definiert werden.
Die explizite angabe des \emph{Context} Receiver Parameters ist nun nicht mehr nötig, da dies durch die Setup-Funktion an innere Funktionen propagiert wird. 

Die Matching Regeln zwischen den verschiedenen BPMN-Events und dem BPMN-Gateway wurden im \cref{sec:matching_model_elements} bereits implementiert.
Für die Erweiterung muss damit nur die Matching Regel für BPMN-Activities mittels Name-Matching hinzugefügt werden.
Dabei wird in den generischen Parametern nach BPMN-Activities und BROS-Events gefiltert.
Diese werden mittels ihrer Namen verglichen und bei erfolgreichen Vergleich zum Matching hinzugefügt.
Da BROS-Events und BROS-ReturnEvents unterschiedliche Metamodell Typen haben, muss diese Matching-Regel ein weiteres mal hinzugefügt werden, wobei der generische Parameter für das BROS-Modell auf ein BROS-ReturnEvent gesetzt wird.
Die Implementierung des Lambdas bleibt unverändert.

\begin{lstlisting}[language=Kotlin, caption=Matching Regel zwischen BPMN-Activities und BROS-Events, label=lst:matching_activity_event]
match<BpmnTask, Event> { bpmn, bros ->
    matchStrings(bpmn.element.name, bros.element.desc)
}
\end{lstlisting}

Nachdem das Matching auf die unterstützten Modellelemente erweitert wurde, kann nun die Verifikationsregel implementiert werden.
Anders als die in \cref{sec:implementaion_consistency_rules} implementierten Regeln ist Regel 6 von dem BROS-Modell aus gerichtet.
Darum wird nicht die Funktion \emph{verifyBpmn} sondern äquivalente Funktion \emph{verifyBros} genutzt.
Im generischen Parameter der Funktion wird nach einem BROS-Event gefiltert.
Innerhalb des Lambdas wird überprüft ob es ein Element im Matching des BROS-Events gibt das ein BPMN-Element ist.
Hier wird zur Vereinfachung des Quellcodes ausgenutzt, das mit den Matching Regeln keine unzulässigen BROS-Elemente im Matching aufgenommen werden können.
Sollte das Matching mit anderen BROS-Event zu BPMN-Element Regeln erweitert werden, muss im Schleifenkörper eine genauere Überprüfung des BPMN-Element Typs erfolgen.
Analog zu der neuen Matching-Regel muss auch die Verifikationsregel ein weiteres mal mit dem Filter nach BROS-ReturnEvents eingefügt werden.

\begin{lstlisting}[language=Kotlin, caption=Implementierung von Regel 6, label=lst:implementation_rule_6]
verifyBros<Event> { bros ->
    for (element in bros.matchingElements) {
        val match = element.element as? BpmnElement ?: continue
        return@verifyBros Result.match("...", bpmn = element)
    }
    Result.error("...")
}
\end{lstlisting}

Ein Nachteil der Kotlin/JS Plattform ist das Fehlen von Reflection und Annotationprcessing wie in Java bzw. Kotlin/JVM.
Aus diesem Grund wird die neu hinzugefügte Setup-Funktion nicht automatisch erkannt und muss manuell registriert werden.
In der Datei \emph{io.framed.modules.Main.kt} existiert eine Liste von Setup-Funktionen die ausgeführt werden.
Um nun die die neue Regel hinzuzufügen wird der Liste ein neuer Eintrag (\emph{::setupRule6BrosEvents}) hinzugefügt.
Der Aufruf \emph{::} auf eine Funktion gibt eine Funktionsreferenz zurück und erlaubt eine spätere Ausführung.
Die vollständige Implementierung der Regel 6 ist unter \cref{lst:Rule6BrosEvents} zu finden.
