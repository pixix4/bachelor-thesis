\chapter{Schluss}

\section{Zusammenfassung}

Ziel dieser Arbeit war die automatisierte Konsistenzprüfung von BPMN- und BROS-Modellen.
Dazu wurden anhand bestehender Arbeiten ein neuer Ansatz entwickelt.
Die Funktionalität dieses Ansatzes wurde anhand einer Referenzimplementierung und einer Fallstudie gezeigt.  

\section{Wissenschaftlicher Beitrag}

\begin{itemize}
    \item \textbf{Nature:} beides
    \item \textbf{Diagrams:} BPMN-BROS
    \item \textbf{Consistency Type:} \emph{Intra-Modell (horizontale) Konsistenz}
    \item \textbf{Intermediate Representation:} Nein
    \item \textbf{Consitency Strategy:} \emph{Monitoring} (Auf einem Regelsatz basierend)
    \item \textbf{Automatable:} gut (H)
    \item \textbf{Extensibility:} gut (H)
\end{itemize}

Der Ansatz und das entwickelte Tool geben dem Modellierer Warnungen und Hinweise die auf mögliche Inkonsistenzen deuten.
Dabei werden explizit keine Handlungsempfehlungen oder Lösungsmöglichkeiten gegeben.
Dies liegt in der alleinigen Verantwortung des Modellierers.
Des Weiteren wird keine syntaktische und semantische Konsistenzprüfung der Einzelmodelle durchgeführt.
Die zu überprüfenden Modelle müssen alleinstehend Konsistent sein.
Nur die Korrektheit des Dateiformats wird indirekt überprüft, da es eine Vorraussetzung zum Parsen des Modelles ist.

\section{Zukünftige Arbeiten}

Eine Abgrenzung dieser Arbeit war, das keine Handlungsempfehlungen oder Lösungsmöglichkeiten bei gefundenen Inkonsistenzen gegeben werden.
In diesem Feld sollte evaluiert werden welche Inkonsistenzmuster häufig auftreten und ob diese eine Vorhersehbare Lösungsmöglichkeit aufweisen.
Auch könnte die Fehlertoleranz des Tools erhöht werden, indem man eine syntaktische und semantische Konsistenzprüfung der Einzelmodelle integriert.
Diese ist momentan noch von externen Programmen abhängig oder aufgabe des Modellierers.
Da externe Programme meist auf diese Aufgabe spezialisiert sind müsste zunächst evaluiert werden ob eine integrierte Lösung auch eine gleichbleibende Erkennungsrate besitzt.
Um die Benutzbarkeit des Tools weiter zu verbessern kann eine Integration in FRaMED.io erfolgen.
Die technische Grundlage ist mit der Referenzimplementierung bereits gegeben.
Hierfür müsste noch evaluiert werden wie das Ergebnis der Analyse in die graphische Darstellung integriert werden kann.
