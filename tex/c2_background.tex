\chapter{Hintergrund}

\section{Business Process Model and Notation}

Die Business Process Model and Notation (BPMN) ist ein Standard für die grafische Beschreibung von Geschäftsprozessen.
Dabei wird das Verhalten eines Systems mit einer an Flussdiagrammen angelehnten Form beschrieben.
Aufgrund der größe des BPMN Standards wird nur ein gekürztes Metamodell auf Basis von \cite{Loja2010} betrachtet.

\section{Compartment Role Object Model}

\section{Business Role-Object Specification}

Der neu entwickelte Ansatz der Business Role-Object Specification (BROS) kombiniert die Vorteile der strukturbasierten Modellierung und der verhaltensbasierten Modellierung.
Als Grundlage für BROS dient die strukturbasierte Modellierungssprache Compartment Object Role Model (CROM).
Diese wird mit Hilfe von Events um den Verhaltensaspekt erweitert.
Mit Ausnahme von RoleConstraints wird das vollständige Metamodell aus \cite{Schoen} betrachtet.
