\chapter{Verwandte Arbeiten}

Der Bereich der Konsistenzprüfung zwischen Strukturbasierten und Verhaltensbasierten Modellen ist seit Jahren ein gut erforschter Bereich.
Ins besondere gibt es ein großes Spektrum an Methoden zum Vergleichen von verschiedenen UML Diagrammen.

\section{Klassifikationsschema}

Die bestehenden Methoden für UML Diagramme wurden von \cite{Usman2008} und \cite{Lucas2009} zusammengestellt.
Beide Arbeiten nutzen ein ähnliches Schema:

\begin{itemize}
    \item \textbf{Nature:} Es beschreibt den Fokus der vergleichbaren Modelle. Mögliche Werte sind strukturbasiert, verhaltensbasiert und beides.
    \item \textbf{Diagrams:} Es beschreibt welche konkreten Modelle von der Methode unterstützt werden. Die Arbeiten von \cite{Usman2008} und \cite{Lucas2009} beziehen sich dabei ausschließlich auf UML Diagramme.
    \item \textbf{Consistency Type:} Es beschreibt welche Arten der Konsistenz von der Methode überprüft werden. Dabei wird hauptsächlich unterschieden zwischen \emph{Inter-Modell (vertikale) Konsistenz} (Konsistenz bei verschiedenen Abstraktionsstufen und gleichem Modelltyp), \emph{Intra-Modell (horizontale) Konsistenz} (Konsistenz bei gleicher Abstraktionsstufe und verschienden Modelltypen)und \emph{Evolutionskonsistenz} (Konsistenz der eines Modelles über verschiedende Entwicklungsstufen). Zusätzlich spezifiziert \cite{Usman2008} noch die \emph{semantische-} und die \emph{syntaktische Konsistenz}. Diese Beziehen sich auf die Konsistenz eines Modelles zu seinem Metamodell. Dies wird für die nachfolgende Arbeit als Vorraussetzung angesehen und nicht näher betrachtet.
    \item \textbf{Intermediate Representation:} Es beschreibt ob die Methode eine temporäre Zwischendarstellung benötigt oder nicht.
    \item \textbf{Consitency Strategy:} Es beschreibt die benutzte Validierungsstrategie. Es werden drei verschiedene Strategien genannt und zwar \emph{Analysis} (Auf einem Algorithmus basierend), \emph{Monitoring} (Auf einem Regelsatz basierend) und \emph{Construction} (Auf der Generierung des zu vergleichenden Modelles basierend).
    \item \textbf{Automatable:} Es beschreibt ob die Methode manuell oder automatisiert von einem Programm durchgeführt werden kann. Mögliche Werte sind gut (H), mittel (M) und schlecht (L).
    \item \textbf{Extensibility:} Es beschreibt wie gut die Methode um weitere Konsistenzregeln erweiterbar ist. Mögliche Werte sind gut (H), mittel (M) und schlecht (L).
\end{itemize}

\section{Aktuelle Verfahren zur Konsistenzprüfung}

Transformation zu \emph{CSP/Object-Z}
\begin{itemize}
    \item Rasch und Wehrheim (\cite{Usman2008})
    \item Kim and Carrington (\cite{Usman2008})
\end{itemize}

Transformation zu \emph{Pertinetze}
\begin{itemize}
    \item Shinkawa (\cite{Usman2008})
    \item Liu et al. (\cite{Usman2008})
    \item Bernardi et al. (\cite{Usman2008})
\end{itemize}

Anwendung von \emph{Description Logic}
\begin{itemize}
    \item Satoh et al. (\cite{Usman2008})
    \item Mens et al. (\cite{Usman2008})
    \item \cite{Simmonds2004}
\end{itemize}

Transformation in gemeinsames Modell
\begin{itemize}
    \item \cite{Egyed2001} (ViewIntegra)
\end{itemize}

Nutzung von UML (zB. \emph{OCL})
\begin{itemize}
    \item Briant et al. (\cite{Usman2008})
    \item Eqyed (\cite{Usman2008})
\end{itemize}

\section{Vergleich der bestehenden Verfahren}

\textit{Vergleich der oben genannten Methoden.}
