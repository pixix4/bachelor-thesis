\section{Mit welchem Aufwand ist dieses Verfahren erweiterbar?}


\begin{frame}{Konsistenzbeziehungen}
    \begin{figure}
      \centering
      \begin{subfigure}{0.3\textwidth}
          \centering
          \begin{adjustbox}{width=0.8\linewidth,center}
            \begin{tikzpicture}
              \node at (0,0) {\begin{tikzpicture}[scale=0.5, every node/.style={scale=0.8},>={Latex[length=1.5mm]}]
  \draw[dashed] (-8,-7) -- (8,7);
  \draw (-6,-5.25) node[anchor=south, rotate=41] {\textbf{BPMN}};
  \draw (-6,-5.25) node[anchor=north, rotate=41] {BROS};

  \draw (-8,6) rectangle (0,1);
  \draw (-7,6) -- (-7,1);
  \draw (-7.5,3.5) node[rotate=90] {Order Pizza};
  \draw (-7,3.5) -- (0,3.5);
  \draw (-6.5,4.75) node[color=unimportant,rotate=90] {A};
  \draw (-6.5,2.25) node[color=unimportant,rotate=90] {B};

  \draw[rounded corners=4.8pt] (0,-0.5) rectangle (8,-6.5);
  \draw (0.2,-0.56) -- (0.2,-6.44);
  \draw (0,-1.5) -- (8,-1.5);
  \draw (4,-1) node {OrderPizza};

  \begin{scope}[color=unimportant]
      \draw[rounded corners=4.8pt] (4.2,-2) rectangle (7.2,-4);
      \draw (4.2,-3) -- (7.2,-3);
      \draw (4.2,-3.5) -- (7.2,-3.5);
      \draw (5.6,-2.5) node {A};

      \draw[rounded corners=4.8pt] (0.8,-4) rectangle (3.8,-6);
      \draw (0.8,-5) -- (3.8,-5);
      \draw (0.8,-5.5) -- (3.8,-5.5);
      \draw (2.4,-4.5) node {B};
  \end{scope}
\end{tikzpicture}
};
              \filldraw[draw=background,fill=background,fill opacity=0.8, draw opacity=0.8] (-4,-3.5) rectangle (4,3.5);
            \end{tikzpicture}
          \end{adjustbox}
          \caption*{\tiny{\textcolor{black!20}{Regel 1: BPMN-Process}}}%
      \end{subfigure}
      \hfill
      \begin{subfigure}{0.3\textwidth}
          \centering
          \begin{adjustbox}{width=0.8\linewidth,center}
            \begin{tikzpicture}
              \node at (0,0) {\begin{tikzpicture}[scale=0.5, every node/.style={scale=0.8},>={Latex[length=1.5mm]}]
  \draw[dashed] (-8,-7) -- (8,7);
  \draw (-6,-5.25) node[anchor=south, rotate=41] {\textbf{BPMN}};
  \draw (-6,-5.25) node[anchor=north, rotate=41] {BROS};

  \draw (-8,6) rectangle (0,1);
  \draw (-7,6) -- (-7,1);
  \draw (-7.5,3.5) node[rotate=90] {Chef};

  \draw[rounded corners=4.8pt,color=unimportant] (0,-0.5) rectangle (8,-6.5);
  \draw[color=unimportant] (0.2,-0.56) -- (0.2,-6.44);
  \draw[color=unimportant] (0,-1.5) -- (8,-1.5);
  \draw[color=unimportant] (4,-1) node {OrderPizza};

  \draw[rounded corners=4.8pt] (2.5,-3) rectangle (5.5,-5);
  \draw (2.5,-4) -- (5.5,-4);
  \draw (2.5,-4.5) -- (5.5,-4.5);
  \draw (4,-3.5) node {Chef};
\end{tikzpicture}
};
              \filldraw[draw=background,fill=background,fill opacity=0.8, draw opacity=0.8] (-4,-3.5) rectangle (4,3.5);
            \end{tikzpicture}
          \end{adjustbox}
          \caption*{\tiny{\textcolor{black!20}{Regel 2: BPMN-Swimlane}}}%
      \end{subfigure}
      \hfill
      \begin{subfigure}{0.3\textwidth}
          \centering
          \begin{adjustbox}{width=0.8\linewidth,center}
            \begin{tikzpicture}
              \node at (0,0) {\begin{tikzpicture}[scale=0.5, every node/.style={scale=0.8},>={Latex[length=1.5mm]}]
  \draw[dashed] (-8,-7) -- (8,7);
  \draw (-6,-5.25) node[anchor=south, rotate=41] {\textbf{BPMN}};
  \draw (-6,-5.25) node[anchor=north, rotate=41] {BROS};

  \draw (-8,6) rectangle (0,1);
  \draw (-7,6) -- (-7,1);
  \draw (-7.5,3.5) node[rotate=90] {Order Pizza};
  \draw (-6.5,4.5) node[rotate=90] {A};
  \draw (-6.5,2) node[rotate=90] {B};
  \draw (-7,3) -- (0,3);

  \begin{scope}[color=unimportant]
      \draw (-5,4.5) circle[radius=0.5];
      \draw (-5,4) node[anchor=north] {Start};
      \draw[->] (-4.5,4.5) -- (-2,4.5);
  \end{scope}
  \fill (-1.5,4.5) circle[radius=0.5];
  \fill[color=background] (-1.5,4.5) circle[radius=0.35];
  \fill (-1.5,4.5) circle[radius=0.2];
  \draw (-1.5,4) node[anchor=north] {End};

  \draw[rounded corners=4.8pt] (0,-0.5) rectangle (8,-6.5);
  \draw (0.2,-0.56) -- (0.2,-6.44);
  \draw (0,-1.5) -- (8,-1.5);
  \draw (4,-1) node {OrderPizza};

  \begin{scope}[color=unimportant]
      \draw[rounded corners=4.8pt] (4.2,-2) rectangle (7.2,-4);
      \draw (4.2,-3) -- (7.2,-3);
      \draw (4.2,-3.5) -- (7.2,-3.5);
      \draw (5.6,-2.5) node {A};

      \draw[rounded corners=4.8pt] (0.8,-4) rectangle (3.8,-6);
      \draw (0.8,-5) -- (3.8,-5);
      \draw (0.8,-5.5) -- (3.8,-5.5);
      \draw (2.4,-4.5) node {B};
  \end{scope}

  \filldraw[fill=background] (0,-4) circle[radius=0.5];
  \draw (0,-4) circle[radius=0.35];
  \draw (0,-4.5) node[anchor=north,fill=background] {End};
\end{tikzpicture}
};
              \filldraw[draw=background,fill=background,fill opacity=0.8, draw opacity=0.8] (-4,-3.5) rectangle (4,3.5);
            \end{tikzpicture}
          \end{adjustbox}
          \caption*{\tiny{\textcolor{black!20}{Regel 3: BPMN-TerminationEvent}}}%
      \end{subfigure}
      \begin{subfigure}{0.3\textwidth}
          \vspace{4pt}
          \centering
          \begin{adjustbox}{width=0.8\linewidth,center}
            \begin{tikzpicture}
              \node at (0,0) {\begin{tikzpicture}[scale=0.5, every node/.style={scale=0.8},>={Latex[length=1.5mm]}]
  \draw[dashed] (-8,-7) -- (8,7);
  \draw (-6,-5.25) node[anchor=south, rotate=41] {\textbf{BPMN}};
  \draw (-6,-5.25) node[anchor=north, rotate=41] {BROS};

  \draw (-8,6) rectangle (0,1);
  \draw (-7,6) -- (-7,1);
  \draw (-7.5,3.5) node[rotate=90] {Chef};

  \begin{scope}[color=unimportant]
      \draw (-5.5,3.5) circle[radius=0.5];
      \draw (-5.5,3) node[anchor=north] {Start};
      \draw[->] (-5,3.5) -- (-2,3.5);
  \end{scope}
  \fill (-1.5,3.5) circle[radius=0.5];
  \fill[color=white] (-1.5,3.5) circle[radius=0.35];
  \draw (-1.5,3) node[anchor=north] {End};

  \draw[rounded corners=4.8pt] (0,-0.5) rectangle (8,-6.5);
  \draw (0.2,-0.56) -- (0.2,-6.44);
  \draw (0,-1.5) -- (8,-1.5);
  \draw (4,-1) node {OrderPizza};

  \draw[rounded corners=4.8pt] (4,-3) rectangle (7,-5);
  \draw (4,-4) -- (7,-4);
  \draw (4,-4.5) -- (7,-4.5);
  \draw (5.5,-3.5) node {Chef};

  \begin{scope}[color=unimportant]
      \draw (1.5,-2.5) circle[radius=0.5];
      \draw (1.5,-3) node[anchor=north] {Start};
      \draw[dashed, ->] (2,-2.5) -- (5.5,-2.5) -- (5.5,-3);
  \end{scope}

  \draw (1.5,-5.5) circle[radius=0.5];
  \draw (1.5,-6) node[anchor=north,fill=white] {End};
  \draw[dashed, ->] (5.5,-5) -- (5.5,-5.5) -- (2,-5.5);
\end{tikzpicture}
};
              \filldraw[draw=background,fill=background,fill opacity=0.8, draw opacity=0.8] (-4,-3.5) rectangle (4,3.5);
            \end{tikzpicture}
          \end{adjustbox}
          \caption*{\tiny{\textcolor{black!20}{Regel 4: BPMN-EndEvent}}}%
      \end{subfigure}
      \hfill
      \begin{subfigure}{0.3\textwidth}
          \vspace{4pt}
          \centering
          \begin{adjustbox}{width=0.8\linewidth,center}
            \begin{tikzpicture}
              \node at (0,0) {\begin{tikzpicture}[scale=0.5, every node/.style={scale=0.8},>={Latex[length=1.5mm]}]
  \draw[dashed] (-8,-7) -- (8,7);
  \draw (-6,-5.25) node[anchor=south, rotate=41] {\textbf{BPMN}};
  \draw (-6,-5.25) node[anchor=north, rotate=41] {BROS};

  \draw (-8,6) rectangle (0,1);
  \draw (-7,6) -- (-7,1);
  \draw (-7.5,3.5) node[rotate=90] {Chef};

  \draw (-5.5,3.5) circle[radius=0.5];
  \draw (-5.5,3) node[anchor=north] {Start};
  \begin{scope}[color=unimportant]
      \draw[->] (-5,3.5) -- (-2,3.5);
      \fill (-1.5,3.5) circle[radius=0.5];
      \fill[color=background] (-1.5,3.5) circle[radius=0.35];
      \draw (-1.5,3) node[anchor=north] {End};
  \end{scope}

  \draw[rounded corners=4.8pt] (0,-0.5) rectangle (8,-6.5);
  \draw (0.2,-0.56) -- (0.2,-6.44);
  \draw (0,-1.5) -- (8,-1.5);
  \draw (4,-1) node {OrderPizza};

  \draw[rounded corners=4.8pt] (4,-3) rectangle (7,-5);
  \draw (4,-4) -- (7,-4);
  \draw (4,-4.5) -- (7,-4.5);
  \draw (5.5,-3.5) node {Chef};

  \draw (1.5,-2.5) circle[radius=0.5];
  \draw (1.5,-3) node[anchor=north] {Start};
  \draw[dashed, ->] (2,-2.5) -- (5.5,-2.5) -- (5.5,-3);

  \begin{scope}[color=unimportant]
      \draw (1.5,-5.5) circle[radius=0.5];
      \draw (1.5,-6) node[anchor=north,fill=background] {End};
      \draw[dashed, ->] (5.5,-5) -- (5.5,-5.5) -- (2,-5.5);
  \end{scope}
\end{tikzpicture}
};
              \filldraw[draw=background,fill=background,fill opacity=0, draw opacity=0] (-4,-3.5) rectangle (4,3.5);
            \end{tikzpicture}
          \end{adjustbox}
          \caption*{\tiny{\textcolor{black!20}{Regel 5: BPMN-StartEvent}}}%
      \end{subfigure}
      \hfill
      \begin{subfigure}{0.3\textwidth}
          \vspace{4pt}
          \centering
          \begin{adjustbox}{width=0.8\linewidth,center}
            \begin{tikzpicture}
              \node at (0,0) {\begin{tikzpicture}[scale=0.5, every node/.style={scale=0.8},>={Latex[length=1.5mm]}]
  \draw[dashed] (-8,-7) -- (8,7);
  \draw (-6,-5.25) node[anchor=south, rotate=41] {BPMN};
  \draw (-6,-5.25) node[anchor=north, rotate=41] {\textbf{BROS}};

  \draw (-8,6) rectangle (0,1);
  \draw (-7,6) -- (-7,1);
  \draw (-7.5,3.5) node[rotate=90] {Chef};

  \draw (-5.8,3.5) circle[radius=0.5];
  \fill (-1.2,3.5) circle[radius=0.5];
  \fill[color=white] (-1.2,3.5) circle[radius=0.35];
  \draw (-4.8,3.5) -- (-4.4,3.9) -- (-4,3.5) -- (-4.4,3.1) -- cycle;
  \draw (-3,3.5) -- (-2.6,3.9) -- (-2.2,3.5) -- (-2.6,3.1) -- cycle;
  \draw[->] (-5.2,3.5) -- (-4.8,3.5);
  \draw[->] (-2.2,3.5) -- (-1.7,3.5);
  \draw[->] (-4.4,3.9) -- (-4.4,4.5) -- (-2.6,4.5) -- (-2.6,3.9);
  \draw (-3.5,4.5) node[anchor=south] {Select A};
  \draw[->] (-4.4,3.1) -- (-4.4,2.5) -- (-2.6,2.5) -- (-2.6,3.1);
  \draw (-3.5,2.5) node[anchor=north] {Select B};

  \draw[rounded corners=4.8pt] (0,-0.5) rectangle (8,-6.5);
  \draw (0.2,-0.56) -- (0.2,-6.44);
  \draw (0,-1.5) -- (8,-1.5);
  \draw (4,-1) node {OrderPizza};

  \draw[rounded corners=4.8pt] (4,-3.5) rectangle (7,-5.5);
  \draw (4,-4.5) -- (7,-4.5);
  \draw (4,-5) -- (7,-5);
  \draw (5.5,-4) node {Chef};

  \draw (1.5,-2.5) circle[radius=0.5];
  \draw (1.8,-3) node[anchor=north] {Select A};
  \draw[dashed, ->] (2,-2.5) -- (5.5,-2.5) -- (5.5,-3.5);
\end{tikzpicture}
};
              \filldraw[draw=background,fill=background,fill opacity=0, draw opacity=0] (-4,-3.5) rectangle (4,3.5);
            \end{tikzpicture}
          \end{adjustbox}
          \caption*{\tiny{Regel 6: BROS-Event}}%
      \end{subfigure}
    \end{figure}
  \end{frame}

\begin{frame}{BROS-Event - BPMN-Element}
    \begin{figure}
      \centering
      \begin{adjustbox}{width=0.5\linewidth,center}
        \begin{tikzpicture}[scale=0.5, every node/.style={scale=0.8},>={Latex[length=1.5mm]}]
  \draw[dashed] (-8,-7) -- (8,7);
  \draw (-6,-5.25) node[anchor=south, rotate=41] {BPMN};
  \draw (-6,-5.25) node[anchor=north, rotate=41] {\textbf{BROS}};

  \draw (-8,6) rectangle (0,1);
  \draw (-7,6) -- (-7,1);
  \draw (-7.5,3.5) node[rotate=90] {Chef};

  \draw (-5.8,3.5) circle[radius=0.5];
  \fill (-1.2,3.5) circle[radius=0.5];
  \fill[color=white] (-1.2,3.5) circle[radius=0.35];
  \draw (-4.8,3.5) -- (-4.4,3.9) -- (-4,3.5) -- (-4.4,3.1) -- cycle;
  \draw (-3,3.5) -- (-2.6,3.9) -- (-2.2,3.5) -- (-2.6,3.1) -- cycle;
  \draw[->] (-5.2,3.5) -- (-4.8,3.5);
  \draw[->] (-2.2,3.5) -- (-1.7,3.5);
  \draw[->] (-4.4,3.9) -- (-4.4,4.5) -- (-2.6,4.5) -- (-2.6,3.9);
  \draw (-3.5,4.5) node[anchor=south] {Select A};
  \draw[->] (-4.4,3.1) -- (-4.4,2.5) -- (-2.6,2.5) -- (-2.6,3.1);
  \draw (-3.5,2.5) node[anchor=north] {Select B};

  \draw[rounded corners=4.8pt] (0,-0.5) rectangle (8,-6.5);
  \draw (0.2,-0.56) -- (0.2,-6.44);
  \draw (0,-1.5) -- (8,-1.5);
  \draw (4,-1) node {OrderPizza};

  \draw[rounded corners=4.8pt] (4,-3.5) rectangle (7,-5.5);
  \draw (4,-4.5) -- (7,-4.5);
  \draw (4,-5) -- (7,-5);
  \draw (5.5,-4) node {Chef};

  \draw (1.5,-2.5) circle[radius=0.5];
  \draw (1.8,-3) node[anchor=north] {Select A};
  \draw[dashed, ->] (2,-2.5) -- (5.5,-2.5) -- (5.5,-3.5);
\end{tikzpicture}

      \end{adjustbox}
    \end{figure}
\end{frame}

\begin{frame}[fragile]{BROS-Event - BPMN-Element}
\begin{lstlisting}[language=Prolog]
rule_6(Bros) :- (bros(Bros, "Event"); bros(Bros, "ReturnEvent")) ->
    (
        bpmn(Bpmn, "StartEvent"), match(Bpmn, Bros);
        bpmn(Bpmn, "EndEvent"), match(Bpmn, Bros);
        bpmn(Bpmn, "TerminationEvent"), match(Bpmn, Bros);
        bpmn(Bpmn, "Event"), match(Bpmn, Bros);
        bpmn(Bpmn, "Activity"), match(Bpmn, Bros);
        bpmn(Bpmn, "Gateway"), match(Bpmn, Bros)
    ).
\end{lstlisting}
\end{frame}

\begin{frame}[fragile]{BROS-Event - BPMN-Element}
\begin{lstlisting}[language=Prolog]
match<BpmnTask, BrosEvent> { bpmn, bros ->
    matchStrings(bpmn.element.name, bros.element.desc)
}
\end{lstlisting}
\end{frame}

\begin{frame}[fragile]{BROS-Event - BPMN-Element}
\begin{lstlisting}[language=Kotlin]
verifyBros<BrosEvent> { bros ->
    for (bpmn in bros.matchingElements) {
        if (bpmn.checkType<BpmnElement>()) {
            return@verifyBros Result.match("...", bpmn = bpmn)
        }
    }
    Result.error("...")
}
\end{lstlisting}
\end{frame}