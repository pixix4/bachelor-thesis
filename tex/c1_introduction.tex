\chapter{Einleitung}
\label{chap:introduction}

\section{Motivation}

Die heutigen Methoden zur Erstellung von Software hängen stark von geeignet definierten Modellen ab, um die Struktur und das Verhalten der Software zu spezifizieren.
Einerseits werden zur Strukturdefinition häufig UML-Strukturdiagramme verwendet, wie z. B. Klassendiagramme oder Komponentendiagramme.
Andererseits werden prozedurale Modelle verwendet, um das Verhalten der Software darzustellen, z. B. BPMN-Diagramme, Sequenzdiagramme oder Petrinetze.
Dennoch besteht eine Lücke zwischen diesen beiden Modellperspektiven: Während die Geschäftsprozesse in prozeduralen Modellen modelliert werden, kann die eigentliche Implementierung der Software nicht ohne die strukturbasierten Modelle erfolgen.
Die strukturbasierten Modelle müssen daher an den verhaltensbasierten Modellen ausgerichtet sein, damit das anschließend entwickelte Softwaresystem auch die in den Vorgehensmodellen definierten Geschäftsprozesse umsetzt.

Derzeit gibt es keine systematische Möglichkeit, prozessuale Geschäftsprozesse, in Form von BPMN-Prozessen, in strukturbasierte, Modellen der Software zu spezifizieren, um eine solche Konsistenz sicherzustellen.
Als erster Ansatz wird dieses Problem in der Sprache der Business Role-Object Specification (BROS) gelöst, indem zeitliche Elemente in eine statische strukturbasierte Modellspezifikation eingefügt werden.
Es ist jedoch eine manuelle, komplexe und fehleranfällige Aufgabe, die Konsistenz von BROS mit einem prozeduralen Geschäftsprozessen sicherzustellen und zu überprüfen.
Daher wird in dieser Arbeit die Konsistenz zwischen BROS und der prozeduralen BPMN untersucht.

\section{Problemdefinition}

Die Überprüfung der Konsistenz zwischen Modellen ist auf dem Gebiet von UML gut erforscht und es existieren viele verschiedene Ansätze.
So kann nicht nur die Konsistenz eines Modelles zu seinen vertiefenden Modellen, sondern auch Konsistenz zwischen mehreren Modellarten überprüft werden.
Dies ermöglicht es die Konsistenz der Modelle fortlaufend, während der gesamten Konzeptions- und Entwicklungsphase eines Projektes, automatisiert zu überprüfen. 
Allerdings gilt dies nicht für die Konsistenz zwischen der neue Modellierungssprache BROS und der BPMN.
Für diese Modelle gibt es noch kein Verfahren oder Ansatz, um die Konsistenz zwischen den Modellen zu überprüfen.
Allerdins existieren für beide Modelle schon Verfahren mit Toolunterstützung um die Konsistenz innerhalb der Modelle zu überprüfen.
Um diese fehlende Konsistenzprüfung zu entwickeln, muss zunächst geklärt werden, welche Konsistenzregeln zwischen BPMN- und BROS-Modellen gelten.
Da solche Konsistenzregeln auf unterschiedlichen Weisen erstellt und formalisiert werden können, soll dabei besonders auf eine mögliche Automatisierung geachtet werden.
Wichtig für diese Arbeit sind die Inkonsistenzen, die potentielle Widersprüche zwischen den Modellen darstellen. 
Die Automatisierung von einem Konsistenzvergleich ist eine wichtiger Aspekt in der Konsistenzprüfung, da eine manuelle Überprüfung nicht nur fehleranfälliger ist, sondern auch besonders bei großen Modellen sehr zeit- und damit auch kostspielig ist.
Sollte eine automatische Überprüfung der Konsistenzregeln möglich sein, soll untersucht werden, wie eine mögliche Implementierung des automatisierten Konsistenzvergleiches realisiert werden kann.
Dazu soll eine Tool entwickelt werden, das mit dem bestehenden BROS-Editor \emph{FRaMED.io} kompatibel ist.
Für das Tool ist es ausreichend die Inkonsistenzen zu finden.
Es ist nicht notwendig eine Handlungsempfehlung für den Modellierer geben oder eine automatische Korrektur durchzuführen. 
Da in dieser Arbeit erstmals die Konsistenz zwischen BPMN und BROS untersucht wird, soll bei der Implementierung besonders auf die Modularität und Erweiterbarkeit um neue Konsistenzaspekte geachtet werden.

\section{Struktur der Arbeit}

Um die Konsistenzbedingungen zwischen BPMN- und BROS-Modellen zu erstellen, werden im \cref{chap:background} zunächst die Modellelemente, die in dieser Arbeit von BPMN und BROS genutzt werden, vorgestellt und jeweils mit einem Metamodell detaillierter beschrieben.
Um einen Einstieg in das Themengebiet des Konsistenzvergleiches zu bekommen werden im \cref{chap:related_work} aktuelle Konsistenzprüfungsverfahren für UML-Modelle analysiert.
Diese werden anhand eines Schemas klassifiziert und im Hinblick auf ihre Anwendbarkeit zu weiteren Modellarten untersucht.
Da BROS eine neue Modellierungssprache ist, existieren noch keine Konsistenzregeln zu anderen Modellen und damit auch nicht zu BPMN.
Daher werden im \cref{chap:consistency} Regeln aufgestellt um die Konsistenz zwischen BPMN- und BROS-Modellen zu gewährleisten.
Zur Erläuterung der Regeln wird jeweils ein Beispiel und eine formale Definition gegeben.
Für die automatisierte Überprüfung der Konsistenzregeln wird eine Referenzarchitektur erstellt und im \cref{chap:implementation} implementiert.
Die Implementierung nutzt als Modellquellen die Editoren \emph{bpmn.io} und \emph{FRaMED.io}.
Anschließend wird sie im \cref{chap:evaluation} anhand eines Beispiel-Use-Case evaluiert.
Zusätzlich wird die Erweiterbarkeit der Implementierung um ein neue Regeln demonstriert.
Abschließend wird im \cref{chap:conclusion} der Erfolg der Arbeit diskutiert und ein Ausblick auf zukünftige Arbeiten auf diesem Gebiet gegeben.
