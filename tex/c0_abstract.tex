\chapter*{Abstract}

Die heutigen Methoden zur Erstellung von Software hängen stark von geeignet definierten Modellen ab, um die Struktur und das Verhalten der Software zu spezifizieren.
Die strukturbasierten Modelle müssen an den verhaltensbasierten Modellen ausgerichtet sein, damit das anschließend entwickelte Softwaresystem auch die in den Vorgehensmodellen definierten Geschäftsprozesse umsetzt.
Derzeit gibt es keine systematische Möglichkeit, Geschäftsprozesse (zum Beispiel in Form von BPMN-Prozessen) in strukturbasierte Modellen der Software zu spezifizieren, um eine solche Konsistenz sicherzustellen.
Als erster Ansatz wird dieses Problem in der Sprache der Business Role-Object Specification (BROS) gelöst, indem zeitliche Elemente in eine statische strukturbasierte Modellspezifikation eingefügt werden.
Es ist jedoch eine manuelle, komplexe und fehleranfällige Aufgabe, die Konsistenz von BROS mit einem bestimmten Geschäftsprozess sicherzustellen und zu überprüfen.

In dieser Arbeit wird die Konsistenz zwischen BROS und der prozeduralen BPMN untersucht.
Zu diesem Zweck werden die Modellelemente in einem BROS- und einem BPMN-Modell miteinander verglichen, um Konsistenzprobleme zu ermitteln.
Konsistenzprobleme treten auf, wenn bestimmte Aspekte der Konsistenz zwischen den Modellarten verletzt werden.
Basierend auf dieser Analyse werden dem Modellierer Warnungen gegeben, wenn Konsistenzprobleme auftreten und was die Ursache dieser Probleme ist.
Diese Aufgabe wird automatisch über ein Tool ausgeführt werden. Die \emph{Proof-of-concept-Implementierung} nutzt dazu die Modelle des BROS-Editor \emph{FRaMED.io} und des BPMN-Editor \emph{bpmn.io}.
