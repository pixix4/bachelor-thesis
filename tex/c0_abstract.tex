\chapter*{Abstract}

\textit{Gekürzte Version der Einleitung}
\vspace{10pt}

Die heutigen Methoden zur Erstellung von Software hängen stark von geeigneten definierten Modellen ab, um die Struktur und das Verhalten der Software zu spezifizieren.
Die Strukturmodelle müssen an den Verhaltensmodellen ausgerichtet sein, damit das anschließend entwickelte Softwaresystem auch die in den Vorgehensmodellen definierten Geschäftsprozesse umsetzt.
Derzeit gibt es keine systematische Möglichkeit, prozessuale Geschäftsprozesse (z.B. in Form von BPMN-Prozessen) in Strukturmodellen der Software zu spezifizieren, um eine solche Konsistenz sicherzustellen.
Als erster Ansatz wird dieses Problem in der Sprache der Business Role-Object Specification (BROS) gelöst, indem zeitliche Elemente in eine statische Strukturmodellspezifikation eingefügt werden.
Es ist jedoch eine manuelle, komplexe und fehleranfällige Aufgabe, die Konsistenz von BROS mit einer bestimmten prozeduralen Geschäftsprozessen sicherzustellen und zu überprüfen.

In dieser Arbeit wird die Konsistenz zwischen BROS und der prozeduralen BPMN untersucht.
Zu diesem Zweck werden die Modellelemente in einem BROS- und einem BPMN-Modell miteinander verglichen, um etwaige Abweichungen in Bezug auf mehrere Konsistenzkonzepte, sogenannte Konsistenzbeschränkungen, zu ermitteln.
Basierend auf dieser Analyse werden dem Modellierer Warnungen gegeben, wenn Konsistenzbeschränkungen verletzt werden und wie die Probleme möglicherweise gelöst werden können.
Diese Aufgabe wird automatisch über ein Tool ausgeführt werden. Die Proof-of-concept Implementierung nutzt dazu die Modelle des BROS-Editor FRaMED.io und des BPMN-Editor bpnm.io.
