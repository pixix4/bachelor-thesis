\section{Wie lassen sich die Konsistenzbedingungen automatisiert überprüfen?}

\begin{frame}{Ablauf der Konsistenzprüfung}
  \begin{figure}[b]
  \centering
  \begin{adjustbox}{width=\linewidth,center}
    \begin{tikzpicture}[scale=0.5]
    
    \draw (-7,3.3) node {\textbf{Modelle\vphantom{g}}};
    \draw (0,3.3) node {\textbf{Matching}};
    \draw (7,3.3) node {\textbf{Verifikation\vphantom{g}}};
    \draw (14,3.3) node {\textbf{Ergebnis}};
    
    \draw (-7.5,2) -- (-7.5,0.5) -- (-6.5,0.5) -- (-6.5,1.7) -- (-6.8,2) -- cycle;
    \draw (-7.2, 1.5) -- (-6.8,1.5);
    \draw (-7.2, 1.25) -- (-6.8,1.25);
    \draw (-7.2, 1) -- (-6.8,1);
    \draw (-7,0.5) node[anchor=north,scale=0.7] {BPMN};
    \draw (-7.5,-0.5) -- (-7.5,-2) -- (-6.5,-2) -- (-6.5,-0.8) -- (-6.8,-0.5) -- cycle;
    \draw (-7.2, -1.5) -- (-6.8,-1.5);
    \draw (-7.2, -1.25) -- (-6.8,-1.25);
    \draw (-7.2, -1) -- (-6.8,-1);
    \draw (-7,-2) node[anchor=north,scale=0.7] {BROS};
    
    
    \draw[-{Triangle[length=5pt]}] (-4,0) -- (-3,0);
    
    
    \draw (0,1.8) node[scale=0.6] {Orakelfunktion};
    \draw (0,1.0) node[scale=0.7] {$$f(x)$$};
    \draw (-1,1.4) rectangle (1,0.6);
    
    \draw[color=layer3,dashed] (-1,-0.15) -- (1,-0.15);
    \draw[color=layer3,dashed] (-1,-0.95) -- (1,-1.75);
    \draw[color=layer3,dashed] (1,-0.95) -- (-1,-1.75);
    
    \draw (-1.75,0.1) rectangle (-1,-0.4);
    \draw (-1.75,-0.7) rectangle (-1,-1.2);
    \draw (-1.75,-1.5) rectangle (-1,-2);
    
    \draw (1.75,0.1) rectangle (1,-0.4);
    \draw (1.75,-0.7) rectangle (1,-1.2);
    \draw (1.75,-1.5) rectangle (1,-2);
    
    
    \draw[-{Triangle[length=5pt]}] (3,0) -- (4,0);
    
    
    \draw (6,1.5) rectangle (8,-1.5);
    \draw (6.5,0.75) node[scale=0.5] {1} (6.7,0.75) -- (7.5,0.75);
    \draw (6.5,0.25) node[scale=0.5] {2} (6.7,0.25) -- (7.5,0.25);
    \draw (6.5,-0.25) node[scale=0.5] {3} (6.7,-0.25) -- (7.5,-0.25);
    \draw (6.5,-0.75) node[scale=0.5] {4} (6.7,-0.75) -- (7.5,-0.75);
    \filldraw[fill=background] (8,-1.5) circle[radius=0.5];
    \draw[line width=1pt] (7.75, -1.5) -- (7.9,-1.7) -- (8.25,-1.3);
    
    
    \draw[-{Triangle[length=5pt]}] (10,0) -- (11,0);
    
    
    \draw (12.75, 1.6) rectangle (14.75,1.1);
    \draw (12.75, 0.7) rectangle (14.75,0.2);
    \draw (12.75, -0.7) rectangle (14.75,-0.2);
    \draw (12.75, -1.6) rectangle (14.75,-1.1);

    \draw[line width=0.7pt] (15, 1.35) -- (15.2,1.18) -- (15.45,1.55);
    \draw[line width=0.7pt] (15, 0.45) -- (15.2,0.28) -- (15.45,0.65);
    \draw[line width=0.7pt] (15.075, -0.3) -- (15.375,-0.6) (15.375, -0.3) -- (15.075,-0.6);
    \draw[line width=0.7pt] (15, -1.35) -- (15.2,-1.52) -- (15.45,-1.15);

    \draw (12.8, -0.5) -- (13.05,-0.25);
    \draw (12.95, -0.65) -- (13.35,-0.25);
    \draw (13.25, -0.65) -- (13.65,-0.25);
    \draw (13.55, -0.65) -- (13.95,-0.25);
    \draw (13.85, -0.65) -- (14.25,-0.25);
    \draw (14.15, -0.65) -- (14.55,-0.25);
    \draw (14.45, -0.65) -- (14.7,-0.4);
    
    
    \draw (-3.5,0.1) -- (-3.5,2.5) -- (10.5,2.5) -- (10.5,0.1);
    \draw (-3.5,-0.1) -- (-3.5,-2.5) -- (10.5,-2.5) -- (10.5,-0.1);
    \end{tikzpicture}%
  \end{adjustbox}
  \label{fig:consistencySymbolicPicture}
\end{figure}
\end{frame}

\begin{frame}{Modelle}
  \begin{figure}
  \centering
  \begin{adjustbox}{width=.75\linewidth,center}
    \begin{tikzpicture}
        \draw[dashed] (0,4.4) -- (0,-5.4);

        \draw (-1,0) rectangle (-8,3);
        \draw (-7,0) -- (-7,3);
        \draw (-7.5,1.5) node[rotate=90] {\textbf{Chef}};
        \draw (-6,1.5) circle [radius=0.5];
        \fill (-2,1.5) circle [radius=0.5];
        \fill[color=background] (-2,1.5) circle [radius=0.4];
        \draw (-6,0.8) node {Start};
        \draw (-2,0.8) node {End};
        \draw[->] (-5.5,1.5) -- (-2.5,1.5);
        \draw (-4.5,4) node {\Large\textbf{{BPMN}}};

        \draw[rounded corners=12pt] (3,0.5) rectangle (6,2.5);
        \draw (3,1.5) -- (6,1.5);
        \draw (3,1) -- (6,1);
        \draw (4.5,2) node {\textbf{Chef}};
        \draw (1.5,1.5) circle [radius=0.5];
        \draw (7.5,1.5) circle [radius=0.5];
        \draw (1.5,0.8) node {Start};
        \draw (7.5,0.8) node {End};
        \draw[dashed,->] (2,1.5) -- (3,1.5);
        \draw[dashed,->] (6,1.5) -- (7,1.5);
        \draw (4.5,4) node {\Large\textbf{{BROS}}};

        \draw[color=layer1] (-4.5,-0.7) node[fill=background] {Process} --
            (-4.5,-2) node[fill=background] {Lane(Chef)} --
            (-2.5,-4) node[fill=background] {EndEvent(End)} (-4.5,-2) --
            (-6.5,-4) node[fill=background] {StartEvent(Start)};
        \draw[color=layer2, ->] (-6,-4.3) -- (-6,-5) -- (-3,-5) -- (-3,-4.3); 
        \draw[color=layer2] (-4.5,-5) node[anchor=north] {\small{SequenceFlow}};

        \draw[color=layer1] (4.5,-0.7) node[fill=background] {Scene} --
            (2,-2) node[fill=background] {RoleType(Chef)} (4.5,-0.7) --
            (4.5,-3) node[fill=background] {Event(Start)} (4.5,-0.7) --
            (7,-4) node[fill=background] {Event(End)};
        \draw[color=layer2, ->] (3.3,-3) -- (2.5,-3) -- (2.5,-2.3); 
        \draw[color=layer2] (2.8,-3) node[anchor=north] {\small{CreateRelation}};
        \draw[color=layer2, ->] (1,-2.3) -- (1,-4) -- (5.3,-4); 
        \draw[color=layer2] (2.8,-4) node[anchor=north] {\small{DestroyRelation}};

        \draw[dash pattern=on 4pt off 2pt, color=layer3] (-3.2,-2) -- (0.3,-2);
        \draw[dash pattern=on 4pt off 2pt, color=layer3] (-5.5,-3.7) -- (-5.5,-3.55) -- (4.5,-3.55) -- (4.5,-3.3);
        \draw[dash pattern=on 4pt off 2pt, color=layer3] (-2,-4.3) -- (-2,-5) -- (7,-5) -- (7,-4.3);

        \draw[color=layer1] (-5, -6) -- (-4,-6) node[anchor=west] {Schicht 1};
        \draw[color=layer2, ->] (-1, -6) -- (0,-6) node[anchor=west] {Schicht 2};
        \draw[color=layer3, dash pattern=on 4pt off 2pt] (3, -6) -- (4,-6) node[anchor=west] {Schicht 3};
    \end{tikzpicture}%
  \end{adjustbox}
  \label{fig:ModelToGraph}
\end{figure}
\end{frame}

\begin{frame}{Matching}
  Matching der Modellelemente anhand des Namen und des Typs

  Algorithmus des Name-Matching:
  \begin{enumerate}
    \item Namen in Teilwörter aufteilen (Anhand von `` '' und Groß/Kleinschreibung)
    \item Endung der Teilwörter entfernen (Letzte 2 Zeichen entfernen)
    \item Alle Teilwörter des kürzeren Namens müssen im längerem Namen enthalten sein
  \end{enumerate}
\end{frame}
\begin{frame}{Matching}
  \begin{figure}
    \centering
    \begin{align}
        \text{'Aktion war erfolgreich'}\ ,&\ \text{'ErfolgreicheAktion'} \nonumber\\
        \text{\{'aktion', 'erfolgreich', 'war'\}}\ ,&\ \text{\{'aktion', 'erfolgreiche'\}} \nonumber\\
        \text{\{'aktion', 'erfolgreiche'\}}\ ,&\ \text{\{'aktion', 'erfolgreich', 'war'\}} \nonumber\\
        \text{\{'\textbf{akti}on'} \subseteq \text{'\textbf{akti}on'\}}\ ,&\ \text{\{'\textbf{erfolgreic}he'} \subseteq \text{'\textbf{erfolgrei}ch'\}} \nonumber
    \end{align}
    \label{eq:name_matching}
  \end{figure}
\end{frame}
\begin{frame}[fragile]{Matching}
\begin{lstlisting}[language=Prolog]
Context.match<BpmnLane, BrosRoleType> { lane, role ->
  matchStrings(lane.element.name, role.element.name)
}
\end{lstlisting}
\end{frame}

\begin{frame}[fragile]{Verifikation}
\begin{lstlisting}[language=Prolog]
rule_2(Bpmn) :- bpmn(Bpmn, "Swimlane") ->
  (
    bros(Bros, "RoleType"), match(Bpmn, Bros)
  ).
\end{lstlisting}
\end{frame}
\begin{frame}[fragile]{Verifikation}
\begin{lstlisting}[language=Kotlin]
Context.verifyBpmn<BpmnLane> { bpmn ->
  for (bros in bpmn.matchingElements) {
    if (bros.checkType<BrosRoleType>()) {
      return Result.match("...", bros = bros)
    }
  }
  return Result.error("...")
}
\end{lstlisting}
\end{frame}


\begin{frame}{Ergebnisse}
  Positive und Negative Konsistenzmeldungen
  \begin{itemize}
    \item Referenz auf Modellelemente
    \item Regel die zur Konsistenzmeldungen geführt hat
    \item Textuelle Beschreibung der Ursache
  \end{itemize}
  \begin{figure}
    \centering
    \begin{adjustbox}{width=1.1\linewidth,center}
      \includegraphics{images/example/error5.png}
    \end{adjustbox}
  \end{figure}
\end{frame}

\section{Demo}