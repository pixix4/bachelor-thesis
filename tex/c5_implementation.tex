\chapter{Implementierung der automatischen Konsistenzprüfung}

Um die Umsetzbarkeit und die Funkionalität dieses Ansatzes zu zeigen wurde eine Referenzimplementierung angefertigt. 

\section{Implementierung der Referenzarchitektur}

Eine Anforderung war eine mögliche Integration in den bestehenden BROS-Editor FRaMED.io.
Durch die Verwendung von Kotlin konnte das Parsing der BROS-JSON Datei direkt übernommen werden.
Das Dateiformat von bpmn.io basiert auf XMI und kann nativ von JS geparsed werden.

\section{Matching der Modelelemente}

Um die Verifizierung der Modelle zu ermöglichen muss zunächst ein dazugehöriges Matching aufgebaut werden.
Dazu werden Elemente mit Kompatiblen Typen anhand ihres Namens zugeordnet.
Zusätzlich kann, aufgrund des Fixpunkt-Algorithmus, ein bestehendes Matching auf andere Elemente übertragen werden.
Sollte dies nicht ausreichen kann der Modellierer explizit ein Predefined-Matching einfügen was ein Element Matching erzwingt oder verbietet.

\textit{Erklärung einiger Matching-Regeln und ihrer Syntax}

\section{Implementierung der Konsistenzregeln}

Dank des bestehenden Matchings lassen sich die genannten Konsistenzregeln einfach implementieren.

\textit{Erklärung einiger Verifikationsregeln und ihrer Syntax}

\section{Benutzerinterface}

Da das entwickelte Tool eine Webanwendung ist kann es in allen moderneren Browsern benutzt werden die JS aktiviert haben.
Das Tool hat ein zweistufiges Interface.
Als erster Schritt müssen die Quelldateien der zu überprüfenden Modelle geladen werden.
Dazu können die Dateien einfach per Drag'n'Drop in das Tool geladen werden.
Das Tool erkennt den Inhalt unabhängig des Namens und läd die entsprechende Datei.
Alternativ kann eine manuelle Dateiauswahl genutzt werden oder der Inhalt der Datei in das Textfeld kopiert werden.
Sobald jeweils ein valides BPMN- und BROS-Modell geladen wurden, wird die Konsistenzprüfung automatisch gestartet.

Die Ergebnisse der Konsistenzprüfung werden unterhalb der Eingabemaske angezeigt.
Zunächst werden statistische Informationen zu den geladen Modellen, des Matchings und der Validierung angezeigt.
Unterhalb dieser Statistiken befindet sich die Liste der Validierungsergebnisse.
Mit hilfe der Tableiste kann das BPMN- oder BROS-Matching oder auch das geladene Predefined-Matching anzeigen.
Das Predefined-Matching kann per Klick auf die Element-IDs direkt bearbeitet werden.
