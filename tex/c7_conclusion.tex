\chapter{Schluss}
\label{chap:conclusion}

In diesem Kapitel soll evaluiert werden in wie weit diese Arbeit die im \cref{chap:introduction} genannten Ziele erfüllt hat.

\section{Zusammenfassung}

Ziel der Arbeit war die Konsistenz zwischen BPMN- und BROS-Modellen zu überprüfen.
Dafür wurden bestehende Konsistenzverfahren für UML-Modelle analysiert und anhand ihrer Eigenschaften klassifiziert.
Darauf aufbauend wurden sechs Regeln für die Konsistenzprüfung aufgestellt und formalisiert.
Für die Automatisierung der Konsistenzprüfung wurde eine Referenzarchitektur erstellt, die die formalen Regeln ausführt.
Um die Funktionalität des Verfahren zu zeigen, wurde ein Tool in Kotlin implementiert um zusätzlich eine mögliche Anbindung an den BROS-Editor FRaMED.io zu erleichtern.
Diese Implementierung wurde anschließend anhand des Beispieles einer Pizzabestellung von \cite{Schoen} getestet indem ein zu dem BPMN-Modell konsistentes BROS-Modell entwickelt wurde.
Außerdem wurde die Erweiterbarkeit dieses Verfahrens Regel evaluiert, indem eine weiter Regel zu zu der Implementierung hinzugefügt wurde. 

\section{Wissenschaftlicher Beitrag}

Das in dieser Arbeit vorgestellte Verfahren gehört zu der Klasse der Verfahren ohne Zwischendarstellung (vgl. \cref{tab:Klassifikationsschema_extended}).
Da es auf dem Auswerten verschiedener Konsistenzregeln basiert, nutzt es die Strategie des Monitorings.
Damit ist der ähnlichste Ansatz im Bezug auf UML-Modelle das Verfahren nach \cite{Egyed2006}.
Beide Verfahren testen die Intra-Modell (horizontale) Konsistenz, sind automatisierbar und wurden bereits evaluiert.
Sie unterscheiden sich neben den unterstützten Modellarten hauptsächlich im Grad der Erweiterbarkeit.
Das Verfahren von \cite{Egyed2006} benutzt zum Konsistenzvergleich Regeln auf basis von OCL.
Damit ist die Erweiterbarkeit um neue Modellarten nicht möglich, wenn diese nicht zum UML-Standard gehören.
Das hier genutzte Verfahren hat auch einen hohen Aufwand, wenn eine neue Modellart hinzugefügt wird.
Allerdings ist die Erweiterbarkeit um neue Regeln mit weniger Aufwand verbunden.
Das das Verfahren direkt auf dem Graphen der Metamodell-Instanz arbeitet können beliebige Regeln leicht erstellt werden.
\cite{Egyed2006} Verfahren ist hingegen auf die Möglichkeiten der OCL beschränkt, was die Erstellung einer neuen Regeln erschweren könnte.

\begin{table}
    \centering
    \begin{tabular}{p{1.58cm} p{1.50cm} p{0.95cm} p{2.2cm} p{1.60cm} p{0.33cm}
        p{0.33cm} p{0.67cm} p{0.72cm} p{0.72cm}}
      &
      \rot{Diagrams} &
      \rot{Consistency} \rot{Type} &
      \rot{Consistency} \rot{Strategy} & 
      \rot{Intermediate} \rot{Representation} & 
      \rot{Case Study} & 
      \rot{Automatable} & 
      \rot{Tool Support} & 
      \rot{Model} \rot{Extensibility} & 
      \rot{Rule} \rot{Extensibility} \\
      \toprule
      Rasch 2003    & CD, SM              & Intra            & Monitoring           & CSP/OZ                      & 1          & H           & 0            & H                   & M                  \\
      \midrule
      Shinkawa 2006 & UCD, CD, SD, AD, SC & Inter            & Analysis             & CPN                         & 0          & H           & 0            & M                   & L                  \\
      \midrule
      Mens 2005     & CD, SD, SC          & All              & Monitoring           & Extended UML                & 1          & H           & 1            & H                   & M                  \\
      \midrule
      Egyed 2001    & CD, OD, SD          & Intra, Inter     & Construction         &                             & 0          & H           & part      & M                   & M                  \\
      \midrule
      Egyed 2006    & CD, SD, SC          & Intra            & Monitoring           &                             & 1          & H           & 1            & L                   & M                  \\
      \midrule
      BROS          & BPMN, BROS          & Intra            & Monitoring           &                             & 1          & H           & 1            & L                   & H                 
    \end{tabular}
    \caption{Einordnung des neuen Verfahrens}%
    \label{tab:Klassifikationsschema_extended}
  \end{table}

\section{Zukünftige Arbeiten}

Eine Abgrenzung dieser Arbeit war, das keine Handlungsempfehlungen oder Lösungsmöglichkeiten bei gefundenen Inkonsistenzen gegeben werden.
In diesem Feld sollte evaluiert werden welche Inkonsistenzmuster häufig auftreten und ob diese eine Vorhersehbare Lösungsmöglichkeit aufweisen.
Auch könnte die Fehlertoleranz des Tools erhöht werden, indem man eine syntaktische und semantische Konsistenzprüfung der Einzelmodelle integriert.
Diese ist momentan noch von externen Programmen abhängig oder aufgabe des Modellierers.
Da externe Programme meist auf diese Aufgabe spezialisiert sind müsste zunächst evaluiert werden ob eine integrierte Lösung auch eine gleichbleibende Erkennungsrate besitzt.
Um die Benutzbarkeit des Tools weiter zu verbessern kann eine Integration in FRaMED.io erfolgen.
Die technische Grundlage ist mit der Referenzimplementierung bereits gegeben.
Hierfür müsste noch evaluiert werden wie das Ergebnis der Analyse in die graphische Darstellung integriert werden kann.
