\RequirePackage[ngerman=ngerman-x-latest]{hyphsubst}
\documentclass[ngerman]{tudscrreprt}
\usepackage[T1]{fontenc}
\usepackage{selinput}
\SelectInputMappings{adieresis={ä},germandbls={ß}}
\usepackage[defaultsans]{opensans}
\usepackage{babel}
\usepackage{csquotes}
\usepackage{graphicx}
\usepackage{import}
\usepackage{microtype}
\usepackage[backend=biber,style=authoryear, maxcitenames=1]{biblatex}
\DefineBibliographyStrings{ngerman}{andothers={et\ al\adddot}} 
\AtEveryCite{%
  \let\parentext=\parentexttrack%
  \let\bibopenparen=\bibopenbracket%
  \let\bibcloseparen=\bibclosebracket}
\addbibresource{sources.bib}
\usepackage[hidelinks]{hyperref}
\usepackage[noabbrev]{cleveref}

\setcounter{tocdepth}{1}
\counterwithout{figure}{section}

\begin{document}
\faculty{Fakultät Informatik}
\institute{Institut für Software- und Multimediatech­nik}
\chair{Professur für Softwaretechno­logie}
\date{13.10.2019}
\author{Lars Westermann}
\title{Evaluierung der Konsistenz zwischen Business Process Modellen und Business Role-Object Spezifikation}
\thesis{Bachelorarbeit}
\supervisor{}
\maketitle

\tableofcontents

\subimport*{tex/}{c0_abstract}

\chapter{Hintergrund}

\begin{itemize}
  \item Einführung in BROS und BPMN
  \item Erläuterung des Konsistenzproblemes
\end{itemize}

\section{Methoden der Konsistenzprüfung}

\begin{itemize}
  \item Related-work (Bezüglich UML)
  \item Abgenzung dieser Arbeit
\end{itemize}

\chapter{Konsistenz zwischen BROS und BPMN}

\begin{itemize}
  \item Einführung in das Pizzabeispiel (fehlerhaftes Beispiel was später erweitert wird)
  \item Definition von Konsistenz
\end{itemize}

\section{Das Metamodell als Vergleichsgrundlage}

\begin{itemize}
  \item Vereinfachte Metamodelle für BROS und BPMN, die für die Konsistenzprüfung notwendig sind
\end{itemize}

\section{Konsistenzregeln}

\begin{itemize}
  \item Iterative Verbesserung des Pizzabeispiels
  \item Vorstellung der implementierten Konsistenzregeln
\end{itemize}

\chapter{Implementierung der automatischen Konsistenzprüfung}

\begin{itemize}
  \item Kotlin, webbasiert
  \item Dateiformat BROS und BPMN
  \item Interne Datenstruktur
\end{itemize}

\section{Matching Algorithmus}

\begin{itemize}
  \item String Matching
  \item Fixpunkt Algorithmus
  \item Predefined matchings
\end{itemize}

\section{Verifizierungsalgorithmus}

\begin{itemize}
  \item Verifizierung auf Grundlage des Matchings
\end{itemize}

\section{Erweiterbarkeit dieses Ansatzes}

\begin{itemize}
  \item Zeigen das eine neue Regel leicht hinzugefügt werden kann
  \item Bspw. jedes BROS Event muss auf eine BPMN Element mappen
\end{itemize}

\chapter{Zusammenfassung}

\begin{itemize}
  \item Zusammenfassung der Arbeit
  \item Ausblick auf mögliche Erweiterungen
\end{itemize}

\printbibliography

\end{document}
