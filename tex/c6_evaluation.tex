\chapter{Fallstudie}

\section{Anwendung am Beispiel einer Pizzabestellung}

Nachdem die verschiedenen Konsistenzregeln aufgestellt und implementiert wurden soll nun die Benutzbarkeit und die Nützlichkeit des Tools gezeigt werden.
Dafür wird der Vorgang einer Pizzabestellung betrachtet und mit Hilfe des Tools evaluiert.
Die benutzten BPMN- und BROS-Modelle basieren auf den Modellen von \cite{Schoen}.

\section{Erweiterbarkeit des Ansatzes}

Bisher wurden nur Regeln vorgestellt, die die Konsistenz von BPMN-Modellen zu BROS-Modellen prüfen.
Da das BROS-Modell gegenüber dem BPMN-Modell angereichert werden kann ist dies auch die häufigste Anwendung.
Allerdings können auch einige Regeln in die andere Richtung überprüft werden.
Um gleichzeitig die Erweiterbarkeit des Ansatzes und der Implementierung zu zeigen wird eine Regel hinzugefügt, die die BROS-Events und die Existenz der dazugehörigen BPMN-Elemente verifiziert.
